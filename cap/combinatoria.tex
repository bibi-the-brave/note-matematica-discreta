\chapter{Combinatoria}
\section{Principi di addizione e moltiplicazione}

\begin{defn}[Principio di addizione]
	Si vuole scegliere un oggetto tra gli elementi di $m$ insiemi \emph{disgiunti}.
	Il primo insieme contiene $r_1$ oggetti, il secondo contiene $r_2$ oggetti, \dots, 
	l'n-esimo contiene $r_m$ oggetti. Il numero di possibili scelte di un oggetto da 
	uno degli $m$ insiemi disgiunti è $r_1 + r_2 + \dots + r_m$.
\end{defn}

\begin{ese}
	In una scuola vengono offerti due corsi opzionali, uno di scacchi e uno di violino.
	Il corso di violino viene frequentato da 40 studenti, quello di scacchi 30.
	Quanti studenti hanno scelto corsi  opzionali?
	
	\emph{Soluzione}:
	Per poter utilizzare il principio di addizione è necessario che gli insiemi siano
	disgiunti ed in questo caso, non si sa se lo sono oppure no perché, alcuni studenti
	potrebbero frequentare sia il corso di scacchi che quello di violino. 
	Per ottenere degli insiemi disgiunti è necessario dividere gli studenti in tre gruppi:
	quelli che frequentano solo il corso di violino, quelli che frequentano solo il corso
	di scacchi e quelli che li frequentano entrambi. Dai dati del problema non è chiaro 
	quanti siano i $k$ studenti che frequentano entrambi i corsi. La cardinalità dei nuovi
	insiemi ottenuti è la seguente:	
	\begin{align*}
		40 - k &= \text{\# studenti che frequentano solamente il corso di violino}\\
		30 - k &= \text{\# studenti che frequentano solamente il corso di scacchi}\\
		k &= \text{\# studenti che frequentano entrambi i corsi}
	\end{align*}
	Per il principio di addizione ci sono quindi $(40 - k) + (30 - k) + k$ studenti che
	hanno scelto di frequentare i corsi opzionali.
	\QEDA
\end{ese}

\begin{defn}[Principio di moltiplicazione]
	Si supponga che un esperimento (un processo/procedura) possa essere suddiviso in
	$m$ ordinate fasi successive con $r_1$ differenti esiti per la prima fase,
	$r_2$ differenti esiti per la seconda fase,\dots, $r_m$ differenti esiti per
	l'm-esima fase. Se il numero di esiti di ciascuna fase è \emph{indipendente} dalle
	scelte fatte durante le fasi precedenti e se \emph{le m-uple dei risultati finali 
	sono tutte differenti tra di loro}, allora in totale, l'esperimento ha 
	$r_1 \cdot r_2 \cdot \ldots \cdot r_m$ differenti esiti.
\end{defn}

\begin{defn}[(Alternativa) principio di moltiplicazione]
	Si effettuano $m$ scelte in modo sequenziale. Se:
	\begin{enumerate}
		\item la prima scelta è tra $r_1$ possibili elementi, la seconda tra $r_2$
		possibili elementi, numero che non dipende dal risultato della prima scelta,
		\dots, la m-esima tra $r_m$ possibili elementi, numero che non dipende dal
		risultato delle $m-1$ scelte precedenti. 
		
		\item m-uple di scelte distinte producono risultati distinti.
	\end{enumerate}
	allora il numero di scelte distinte è $r_1 \cdot r_2 \cdot \ldots \cdot r_m$.
\end{defn}

\begin{ese}
	Si lanciano due dadi, uno verde ed uno rosso poi si osserva l'esito dell'esperimento.
	\begin{enumerate}
		\item Quanti differenti esiti possono esserci?
		\item Quanti possono essere i differenti esiti relativi all'uscita di due facce diverse?
	\end{enumerate}
	\emph{Soluzione}:
	\begin{enumerate}
		\item Ogni dado ha sei facce distinte. La prima parte dell'esperimento consiste nel lanciare
		un dado, la seconda nel lanciare l'altro. Per il principio di moltiplicazione vi sono quindi
		$6 \cdot6 = 36$ possibili esiti per questo esperimento.
		\item Si utilizza ancora il principio di moltiplicazione: nella prima fase della procedura
		viene lanciato un dado in cui possono comparire tutte le 6 facce. La seconda fase della
		procedura, indipendentemente dal particolare valore del primo dado può avere come valori
		solamente cinque facce.% che non sono già uscite\dots
		Per questo motivo ci sono $6 \cdot 5 = 30$ esiti. Si noti che per poter applicare il
		principio di moltiplicazione il vincolo ''due facce diverse'' deve essere trasformato
		in ''il valore del primo dado lanciato deve essere differente da quello del secondo''.
		(Una soluzione alternativa che non applica il principio è la seguente: poiché un dado ha sei 
		facce distinte, ci sono solamente sei casi in cui possono uscire
		due dadi con la stessa faccia. Per questo motivo numero di esiti in cui i due dadi hanno facce
		distinte sono $36 - 6 = 30$). \QEDA
	\end{enumerate}
\end{ese}

Quando si cerca di risolvere un problema di combinatoria solitamente si cerca di suddividerlo
in un moderato numero sottoproblemi di più facile risoluzione. Potrebbero esserci modi più
intelligenti per risolverlo ma, se si riesce a ridurlo ad un insieme di sottoproblemi di cui
si ha più familiarità, allora è meno probabile compiere degli errori. 
\begin{ese}
	In uno scaffale di una biblioteca ci sono 11 libri in inglese (distinti), 7 in francese (distinti)
	e 4 in russo (distinti). In quanti modi si possono scegliere una coppia (non ordinata) di libri
	che non siano della stessa lingua?
	
	\emph{Soluzione}:
	Per il principio di moltiplicazione le coppie non ordinate di libri di inglese e francese
	ottenibili sono $11 \cdot 7 = 77$, quelle di inglese e russo sono $11 \cdot 4 = 44$, quelle
	di francese e russo sono $7 \cdot 4 = 28$.
	Questi tre principi di scelte sono tra loro disgiunti e quindi per il principio di addizione 
	ci sono $77 + 44 + 28 = 149$ diversi modi di scegliere.
	\QEDA
\end{ese}

\begin{ese}
	Si hanno disposizione le lettere \emph{a, b, c, d, e, f}, per formare sequenze di lunghezza 3.
	Quante se ne possono formare con le seguenti regole?
	\begin{enumerate}
		\item \textbf{Le ripetizioni sono ammesse}
		\item \textbf{Le ripetizioni non sono ammesse}
		\item \textbf{Le ripetizioni non sono ammesse ed è presente la lettera \emph{e}}
		\item \textbf{Le ripetizioni sono ammesse ed è presente la lettera \emph{e}}
	\end{enumerate}
	\emph{Soluzione}:
	\begin{enumerate}
		\item Con le ripetizioni ci sono $6$ scelte per ciascun carattere della sequenza. 
		Per il principio della moltiplicazione ci sono $6 \cdot 6 \cdot  6 = 216$ sequenze di
		tre lettere con ripetizione.
		\item Senza ripetizioni ci sono $6$ scelte per la prima lettera, $5$ per le rimanenti
		per la seconda, $4$ per la terza. Per il principio di moltiplicazione ci sono 
		$6 \cdot 5 \cdot 4 = 120$ sequenze di tre lettere senza ripetizione.
		\item Se la sequenza deve contenere la lettera \emph{e}, allora ci sono 3 scelte relative
		alla posizione di \emph{e} nella sequenza:
		\begin{center}
			\centering
			\underline{e} \underline{ } \underline{ }
			\quad
			\underline{ } \underline{e} \underline{ }
			\quad
			\underline{ } \underline{ } \underline{e}
		\end{center}
		Come si può vedere, in ciascun diagramma rimangono $5$ scelte per la prima posizione non
		occupata dalla lettera \emph{e}, poi ne rimangono $4$ per la seconda posizione non occupata.
		Per il principio di addizione e per quello della moltiplicazione si hanno in totale 
		$ 5 \cdot 4 +  5 \cdot 4 +  5 \cdot 4 = 60$ scelte.
		\item \dots
		\begin{center}
			\centering
			\underline{e} \underline{ } \underline{ }
			\quad
			\underline{\bcancel{e}} \underline{e} \underline{ }
			\quad
			\underline{\bcancel{e}} \underline{\bcancel{e}} \underline{e}
		\end{center}
	\end{enumerate}
	\dots
	\QEDA
\end{ese}
\section{Permutazioni e disposizioni semplici}