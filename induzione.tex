\chapter{Principio di Induzione}
%Le prove per induzione matematica sono composte da due parti: la prima è il \emph{caso base},
%la seconda è il \emph{passo induttivo}.
%Il \textbf{caso base} dimostra che una proposizione è valida per il primo numero

\begin{defn}[Principio di Induzione (versione 1)]
Sia $P(n)$ una proprietà definita per i numeri natuali n, sia $a$ un intero fissato.
Supponendo che le seguenti due affermazioni siano vere:
\begin{enumerate}
	\item $P(a)$ è vero
	\item $\forall k \in \mathbb{N}$ tale che $k \geqslant a$, se $P(k)$ è vero allora
	$P(k+1)$ è vero.
\end{enumerate}
allora se $ P(n)$ è vera $\forall n \geqslant a$ .
\end{defn}