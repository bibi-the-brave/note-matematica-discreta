\documentclass[a4paper, openany]{book}


\usepackage[utf8]{inputenc}
\usepackage[T1]{fontenc}
\usepackage[italian]{babel}
\usepackage{concrete}
\usepackage{beton}
%\usepackage{concmath}
%\usepackage{ccfonts}
\renewcommand{\bfdefault}{sbc} 

\usepackage{amsmath}
\usepackage{amssymb}
\usepackage{amsfonts}
\usepackage{euler}		%font 'ams euler', lo stesso di 'Concrete Mathematics' di Knuth
\usepackage{amsthm}
\usepackage{mathtools}
\usepackage{float}

\usepackage{tikz}
\usetikzlibrary{arrows.meta}
\usetikzlibrary{graphs}
\usetikzlibrary{arrows}

\theoremstyle{plain} 
\newtheorem{thm}{Teorema}[section] 
\newtheorem{cor}[thm]{Corollario} 
\newtheorem{lem}[thm]{Lemma} 
\newtheorem{prop}[thm]{Proposizione} 

\theoremstyle{definition} 
\newtheorem{defn}{Definizione}[chapter] 
%\theoremstyle{remarks} 
\newtheorem{ese}{Esempio}[section]
\newtheorem{oss}{Osservazione}[section]
\newtheorem{eser}{Esercizio}[section]

\pagestyle{headings}

%\newcommand*{\QEDA}{\hfill\ensuremath{\blacksquare}}%
%\newcommand*{\QEDB}{\hfill\ensuremath{\square}}%

%inizio del documento
\begin{document}

\title{Matematica Discreta\\Laurea in Informatica}
\author{Andrea Favero}
\date{Luglio 2016}
\maketitle
%fine copertina


\tableofcontents    %indice analitico


%primo capitolo
\chapter{Grafi}

\section{Grafi non orientati}

\begin{defn}[grafo non orientato semplice]	
Un \emph{grafo non orientato semplice} $G$ è una coppia ordinata $(V,E)$ dove: $V=\{v_1,\dots,v_n\}$ è
un insieme finito di \emph{vertici} (o \emph{nodi}) ed $E$ è un insieme di coppie
\emph{non ordinate} di vertici 
(\emph{spigoli}\footnote{L'insieme si chiama E perchè in inglese gli spigoli 
sono denominati \emph{edges}}\footnote{ In molti libri di testo $E$, viene rappresentato come
${E = \{ \text{ } \{v_1,v_3\},\text{ } \{v_1,v_7\}, \text{ } \dots \}}$ 
perché non c'è alcun ordine tra gli spigoli.}).
Il grafo è detto \emph{semplice} perché non può avere né cappi né spigoli paralleli.
\end{defn}

\begin{ese}
\[V = \{ v_1, \dots, v_8 \}\]
\[E = \{ (v_1,v_3), (v_1,v_7), (v_2,v_3), (v_4,v_7),(v_6,v_5), (v_4,v_2), (v_3,v_5), (v_5,v_7) \}\]
\end{ese}

\begin{figure}[!ht]
    \centering
    \begin{tikzpicture}
        \node[shape=circle,draw=black] (A) at (0,6) {$v_1$};
        \node[shape=circle,draw=black] (B) at (0,3) {$v_2$};
        \node[shape=circle,draw=black] (C) at (1,4) {$v_3$};
        \node[shape=circle,draw=black] (D) at (2.5,1) {$v_4$};
        \node[shape=circle,draw=black] (E) at (2.5,5) {$v_5$};
        \node[shape=circle,draw=black] (F) at (2,3) {$v_6$};
        \node[shape=circle,draw=black] (G) at (5,6) {$v_7$};
        \node[shape=circle,draw=black] (H) at (5,1) {$v_8$};

        %\path [->] (A) edge node[left] {$5$} (B);
        \path (A) edge (C);
        \path (A) edge (G);
        \path (B) edge (C);
        \path (D) edge (G);
        \path (F) edge (E);
        \path (D) edge (B);
        \path (C) edge (E);
        \path (E) edge (G);
    \end{tikzpicture}
    \caption{un grafo semplice non orientato}
\end{figure}


Lo spigolo $(v_1,v_3)$ ha come \emph{estremi} i vertici $v_1$ e $v_3$.

\begin{ese}
    \begin{figure}[!ht]
        \centering
        \begin{tikzpicture}
            \node[shape=circle,draw=black] (A) at (0,0) {$v_1$};
            \node[shape=circle,draw=black] (B) at (4,0) {$v_2$};

            \path (A) edge [loop] node {} (A);
            \path (A) edge [bend right=20] (B);
            \path (B) edge [bend right=20] (A);
        \end{tikzpicture}
        \caption{un grafo non orientato che non è semplice} \label{fig:no_semplice}
    \end{figure}
Il grafo non orientato in figura~\ref{fig:no_semplice} non è semplice poiché presenta un
cappio sul vertice $v_1$ e ci sono due spigoli paralleli tra i vertici $v_1$ e $v_2$.
\end{ese}

\begin{defn}
Uno spigolo è detto \emph{incidente} nei suoi estremi ed i suoi vertici sono detti \emph{adiacenti}.
\end{defn}

\begin{defn}[cammino]
Un \emph{cammino} è una sequenza di vertici \emph{distinti} in cui ogni coppia di vertici consecutivi
è uno spigolo.
\end{defn}

\begin{ese}
$v_4 \text{ - } v_7 \text{ - } v_5 \text{ - } v_3 \text{ - } v_2 $ è un cammino. Un'altra notazione
per indicare il cammino è: $(v_4, v_7, v_5, v_3, v_2)$.
    \begin{figure}[!ht]
        \centering
        \begin{tikzpicture}
            \node[shape=circle,draw=black] (A) at (0,6) {$v_1$};
            \node[shape=circle,draw=orange] (B) at (0,3) {$v_2$};
            \node[shape=circle,draw=orange] (C) at (1,4) {$v_3$};
            \node[shape=circle,draw=orange] (D) at (2.5,1) {$v_4$};
            \node[shape=circle,draw=orange] (E) at (2.5,5) {$v_5$};
            \node[shape=circle,draw=black] (F) at (2,3) {$v_6$};
            \node[shape=circle,draw=orange] (G) at (5,6) {$v_7$};
            \node[shape=circle,draw=black] (H) at (5,1) {$v_8$};

            %\path [->] (A) edge node[left] {$5$} (B);
            \path (A) edge (C);
            \path (A) edge (G);
            \path[draw=orange,line width=0.1em] (B) edge (C);
            \path[draw=orange,line width=0.1em] (D) edge (G);
            \path (F) edge (E);
            \path (D) edge (B);
            \path[draw=orange,line width=0.1em] (C) edge (E);
            \path[draw=orange,line width=0.1em] (E) edge (G);
        \end{tikzpicture}
        \caption{un cammino $v_4 \text{ - } v_7 \text{ - } v_5 \text{ - } v_3 \text{ - } v_2 $}
    \end{figure}
\end{ese}


\begin{defn}[circuito]
Cammino nel quale il primo e l'ultimo vertice sono adiacenti.
\end{defn}
\begin{ese}
Il grafo in figura~\ref{fig:circ_semplice} ha il circuito:
$v_4 \text{ - } v_7 \text{ - } v_5 \text{ - } v_3 \text{ - } v_2 \text{ - } v_4$.
    \begin{figure}[!h]
    \centering
        \begin{tikzpicture}
            \node[shape=circle,draw=black]  (A) at (0,6) {$v_1$};
            \node[shape=circle,draw=orange] (B) at (0,3) {$v_2$};
            \node[shape=circle,draw=orange] (C) at (1,4) {$v_3$};
            \node[shape=circle,draw=orange] (D) at (2.5,1) {$v_4$};
            \node[shape=circle,draw=orange] (E) at (2.5,5) {$v_5$};
            \node[shape=circle,draw=black]  (F) at (2,3) {$v_6$};
            \node[shape=circle,draw=orange] (G) at (5,6) {$v_7$};
            \node[shape=circle,draw=black]  (H) at (5,1) {$v_8$};

            %\path [->] (A) edge node[left] {$5$} (B);
            \path (A) edge (C);
            \path (A) edge (G);
            \path[draw=orange,line width=0.1em] (B) edge (C);
            \path[draw=orange,line width=0.1em] (D) edge (G);
            \path (F) edge (E);
            \path[draw=orange,line width=0.1em] (D) edge (B);
            \path[draw=orange,line width=0.1em] (C) edge (E);
            \path[draw=orange,line width=0.1em] (E) edge (G);
        \end{tikzpicture}
        \caption{ un circuito 
            $v_4 \text{ - } v_7 \text{ - } v_5 \text{ - } v_3 \text{ - } v_2 \text{ - } v_4$}
        \label{fig:circ_semplice}
    \end{figure}
\end{ese}

\begin{defn}[grafo connesso]
Un grafo si dice \emph{connesso} se per ogni coppia di vertici esiste un cammino
che li collega, altrimenti si dice \emph{disconnesso}.
\end{defn}

\begin{defn}[grafo completo]
Un grafo è \emph{completo} se per ogni coppia di vertici esiste uno spigolo che li collega.
Se il grafo ha $n$ vertici allora è un grafo~$k_n$.
\end{defn}
\begin{ese}
    3 grafi completi sono nella figura \ref{fig:completi}

    \marginpar{
        Nell'ordine:
        \begin{itemize}
            \item un grafo $k_3$
            \item un grafo $k_4$
            \item un grafo $k_5$
        \end{itemize}
    }
    \begin{figure}[!ht]
    \centering
        \begin{tikzpicture}
            \node[shape=circle,draw=black] (A) at (0,0) {$v_1$};
            \node[shape=circle,draw=black] (B) at (3,0) {$v_2$};
            \node[shape=circle,draw=black] (C) at (3,3) {$v_3$};

            \path (A) edge node {} (B);
            \path (A) edge node {} (C);
            \path (B) edge node {} (C);
        \end{tikzpicture}
        \hspace{1cm}
        \begin{tikzpicture}
            \node[shape=circle,draw=black] (A) at (0,0) {$v_1$};
            \node[shape=circle,draw=black] (B) at (3,0) {$v_2$};
            \node[shape=circle,draw=black] (C) at (3,3) {$v_3$};
            \node[shape=circle,draw=black] (D) at (0,3) {$v_4$};

            \path (A) edge node {} (B);
            \path (A) edge node {} (C);
            \path (B) edge node {} (C);
            \path (A) edge node {} (D);
            \path (C) edge node {} (D);
            \path (B) edge node {} (D);

        \end{tikzpicture}
        \hspace{1cm}
        \begin{tikzpicture}
            \node[shape=circle,draw=black] (A) at (0,1.75) {$v_1$};
            \node[shape=circle,draw=black] (B) at (0.75,0) {$v_2$};
            \node[shape=circle,draw=black] (C) at (2.25,0) {$v_3$};
            \node[shape=circle,draw=black] (D) at (3,1.75) {$v_4$};
            \node[shape=circle,draw=black] (E) at (1.5,3) {$v_5$};

            \path (A) edge node {} (B);
            \path (A) edge node {} (C);
            \path (A) edge node {} (D);
            \path (A) edge node {} (E);
            \path (B) edge node {} (C);
            \path (B) edge node {} (D);
            \path (B) edge node {} (E);
            \path (C) edge node {} (D);
            \path (C) edge node {} (E);
            \path (D) edge node {} (E);
        \end{tikzpicture}
        \caption{un $k_3$, $k_4$ e $k_5$} \label{fig:completi}
    \end{figure}
\end{ese}

\begin{defn}[grafo bipartito]
Un grafo è \emph{bipartito}\footnote{I grafi bipartiti saranno discussi in seguito}
se i suoi vertici sono partizionati in due sottoinsiemi, $U$ e $V$,
e ogni spigolo è incidente in un vertice di $U$ e uno di $V$.
\end{defn}

\begin{ese}
Due grafi bipartiti sono in figura \ref{fig:bipartiti}.
    \begin{figure}[!ht]
    \centering
        \begin{tikzpicture}
            \node[shape=circle,draw=black] (A) at (0,3) {$u_1$};
            \node[shape=circle,draw=black] (B) at (3,3) {$w_1$};
            \node[shape=circle,draw=black] (C) at (0,1.5) {$u_2$};
            \node[shape=circle,draw=black] (D) at (3,1.5) {$w_2$};
            \node[shape=circle,draw=black] (E) at (0,0) {$u_3$};
            \node[shape=circle,draw=black] (F) at (3,0) {$w_3$};

            \path (A) edge node {} (B);
            \path (A) edge node {} (F);
            \path (B) edge node {} (C);
            \path (C) edge node {} (D);
            \path (D) edge node {} (E);

        \end{tikzpicture}
        \hspace{1cm}
        \begin{tikzpicture}
            \node[shape=circle,draw=black] (A) at (1.5, 3) {$u_1$};
            \node[shape=circle,draw=black] (B) at (3, 2.25) {$w_1$};
            \node[shape=circle,draw=black] (C) at (3, 0.75) {$u_2$};
            \node[shape=circle,draw=black] (D) at (0, 2.25) {$w_2$};
            \node[shape=circle,draw=black] (E) at (0, 0.75) {$u_3$};
            \node[shape=circle,draw=black] (F) at (1.5, 0) {$w_3$};

            \path (A) edge node {} (D);
            \path (A) edge node {} (B);
            \path (B) edge node {} (C);
            \path (F) edge node {} (E);
            \path (F) edge node {} (C);
            \path (B) edge node {} (C);
            \path (D) edge node {} (E);
        \end{tikzpicture}
        \caption{2 grafi bipartiti} \label{fig:bipartiti}
    \end{figure}
\end{ese}


\section{Grafi orientati}
\begin{defn}[grafo orientato semplice]
Un \emph{grafo orientato semplice} $G$ è una coppia ordinata $(V, A)$ dove:
$V = \{ v_1, \dots, v_n \}$ è un insieme finito di vertici (o nodi) ed
$A$ è un insieme di \emph{coppie ordinate} di vertici dette \emph{archi}.
\end{defn}

\begin{ese}
\[G(V,A) \text{ con } V = \{ v_1, v_2, v_3, v_4, v_5 \} \text{ e} \]
\[A = \{ (v_1, v_3), (v_2, v_1), (v_2, v_5), (v_3, v_5), (v_4, v_1), (v_4, v_3), (v_5, v_2) \} \]
    \begin{figure}[!ht]
    \centering
        \begin{tikzpicture}
            \node[shape=circle,draw=black] (A) at (0,0) {$v_1$};
            \node[shape=circle,draw=black] (B) at (4,0) {$v_2$};
            \node[shape=circle,draw=black] (C) at (4,4) {$v_3$};
            \node[shape=circle,draw=black] (D) at (1,3) {$v_4$};
            \node[shape=circle,draw=black] (E) at (2.5,1.5) {$v_5$};

            \path [->] (A) edge node[left] {} (C);
            \path [->] (B) edge node[left] {} (A);
            \path [->] (B) edge node[left] {} (E);
            \path [->] (C) edge node[left] {} (E);
            \path [->] (D) edge node[left] {} (A);
            \path [->] (D) edge node[left] {} (C);
            \path [->] (E) edge [out=35,in=-2] node[left] {} (B);
        \end{tikzpicture}
    \end{figure}
\end{ese}
Il grafo è semplice perché non ha né cappi né archi paralleli. Essendo orientato
il grafo ha un \emph{nodo iniziale} (\emph{testa}) ed un \emph{nodo finale} (\emph{coda}).

\begin{ese}
\[ G(V, A) \quad V = \{v_1, v_2\}, \quad A = \{ (v_1, v_2), (v_2, v_1) \}\]
$G$ è un grafo orientato. L'arco $(v_1, v_2) \in A$ ha $v_1$ come nodo iniziale
e $v_2$ come finale.
    \begin{figure}[!ht]
    \centering
        \begin{tikzpicture}
            \node[shape=circle,draw=black] (A) at (0, 0) {$v_1$};
            \node[shape=circle,draw=black] (B) at (3, 0) {$v_2$};

            \path [->] (A) edge [bend right=20] node[right] {} (B);
            \path [->] (B) edge [bend right=20] node[left] {} (A);
        \end{tikzpicture}
    \end{figure}
\end{ese}

\begin{ese}
L'immagine in figura~\ref{fig:no_orientato_semplice} non rappresenta un grafo orientato semplice
perchè i vertici $v_1$ e $v_2$ sono collegati da due archi paralleli 
(ovvero due archi che hanno lo stesso nodo iniziale ed anche quello finale), 
inoltre c'è un cappio su $v_1$.
    \begin{figure}[!ht] 
    \centering
        \begin{tikzpicture}
            \node[shape=circle,draw=black] (A) at (0, 0) {$v_1$};
            \node[shape=circle,draw=black] (B) at (3, 0) {$v_2$};

            \path (A) edge [loop above] node {} (A);
            \path [->] (A) edge [bend right=20] node[right] {} (B);
            \path [->] (A) edge [bend left=20] node[right] {} (B);
        \end{tikzpicture}
        \caption{grafo orientato che non è semplice} \label{fig:no_orientato_semplice}
    \end{figure}
\end{ese}

\begin{defn}[cammino orientato]
Un \emph{cammino orientato} è una sequenza di nodi distinti dove, ogni coppia di nodi
consecutivi nel cammino è collegata da un arco.
\end{defn}

\begin{ese}
Nel grafo in figura~\ref{fig:cammino_orientato} è presente il cammino orientato: 
\[v_1 - v_3 - v_5 - v_2\]
    \begin{figure}[!ht]
    \centering
        \begin{tikzpicture}
            \node[shape=circle,draw=orange] (A) at (0,0) {$v_1$};
            \node[shape=circle,draw=orange] (B) at (4,0) {$v_2$};
            \node[shape=circle,draw=orange] (C) at (4,4) {$v_3$};
            \node[shape=circle,draw=black] (D) at (1,3) {$v_4$};
            \node[shape=circle,draw=orange] (E) at (2.5,1.5) {$v_5$};

            \path [->] (B) edge node[left] {} (A);
            \path [draw=orange,line width=0.1em][->] (A) edge node[left] {} (C);
            \path [->] (B) edge node[left] {} (E);
            \path [draw=orange,line width=0.1em][->] (C) edge node[left] {} (E);
            \path [->] (D) edge node[left] {} (A);
            \path [->] (D) edge node[left] {} (C);
            \path [draw=orange,line width=0.1em][->] (E) edge [out=35,in=-2] node[left] {} (B);
        \end{tikzpicture}
        \caption{cammino orientato} \label{fig:cammino_orientato}
    \end{figure}
\end{ese}

\begin{defn}[circuito orientato]
Cammino orientato nel quale esiste un arco dal primo all'ultimo nodo.
\end{defn}

\begin{ese}
In figura~\ref{fig:c_or} è rappresentato il circuito orientato \[v_1 - v_3 - v_5 - v_2 - v_1\]
    \begin{figure}[!ht]
    \centering
        \begin{tikzpicture}
            \node[shape=circle,draw=orange] (A) at (0,0) {$v_1$};
            \node[shape=circle,draw=orange] (B) at (4,0) {$v_2$};
            \node[shape=circle,draw=orange] (C) at (4,4) {$v_3$};
            \node[shape=circle,draw=black] (D) at (1,3) {$v_4$};
            \node[shape=circle,draw=orange] (E) at (2.5,1.5) {$v_5$};

            \path [draw=orange,line width=0.1em][->] (B) edge node[left] {} (A);
            \path [draw=orange,line width=0.1em][->] (A) edge node[left] {} (C);
            \path [->] (B) edge node[left] {} (E);
            \path [draw=orange,line width=0.1em][->] (C) edge node[left] {} (E);
            \path [->] (D) edge node[left] {} (A);
            \path [->] (D) edge node[left] {} (C);
            \path [draw=orange,line width=0.1em][->] (E) edge [out=35,in=-2] node[left] {} (B);
        \end{tikzpicture}
        \caption{circuito orientato} \label{fig:c_or}
    \end{figure}
\end{ese}

\begin{defn}[grafo fortemente connesso]
Per ogni coppia di nodi esiste un cammino orientato che li collega. 
\end{defn}

\begin{ese}
	\[ G( V = \{ v_1, v_2, v_3 \}, \
    A = \{ (v_2, v_1), (v_1, v_3), (v_3, v_1),  (v_3, v_2) \}) \]
    è un grafo fortemente connesso ed è in figura~\ref{fig:forte_connesso}
    \begin{figure}[H]
        \centering
            \begin{tikzpicture}
                \node[shape=circle,draw=black] (A) at (0,0) {$v_1$};
                \node[shape=circle,draw=black] (B) at (2,0) {$v_2$};
                \node[shape=circle,draw=black] (C) at (2,2) {$v_3$};

                %\path [->] (A) edge [bend left=20] node[right] {} (B);
                \path [->] (B) edge [bend left=20] node[right] {} (A);
                \path [->] (A) edge [bend left=20] node[right] {} (C);
                \path [->] (C) edge [bend left=15] node[right] {} (A);
                %\path [->] (B) edge [bend right=20] node[right] {} (C);
                \path [->] (C) edge [bend left=20] node[right] {} (B);%right=20] node[right] {} (B);
            \end{tikzpicture}
            \caption{grafo fortemente connesso} \label{fig:forte_connesso}
    \end{figure}
\end{ese}


\section{Prime proprietà dei grafi non orientati}

Sia $G(V,E)$ un grafo non orientato semplice.
Non è difficile notare che il minimo numero di spigoli che un grafo può avere è $0$ (ogni vertice è isolato) mentre il massimo è:
\marginpar{$|V|$ ed $|E|$ indicano la \emph{cardinalità} di $A$ ed $E$.
La cardinalità di un insieme finito è un numero naturale che rappresenta la quantità di elementi
che costituiscono l'insieme.} \[ \frac{|V|\text{ }(|V|-1)}{2} \]
\begin{ese} grafi con $|V| = 3$ e $|V| = 4$.
\begin{figure}[!ht]		
    \centering
        \begin{tikzpicture}
            \node[shape=circle,draw=black] (A) at (0,0) {$v_1$};
            \node[shape=circle,draw=black] (B) at (3,0) {$v_2$};
            \node[shape=circle,draw=black] (C) at (3,3) {$v_3$};

            \path (A) edge node {} (B);
            \path (A) edge node {} (C);
            \path (B) edge node {} (C);
        \end{tikzpicture}
        \hspace{1cm}
        \begin{tikzpicture}
            \node[shape=circle,draw=black] (A) at (0,0) {$v_1$};
            \node[shape=circle,draw=black] (B) at (3,0) {$v_2$};
            \node[shape=circle,draw=black] (C) at (3,3) {$v_3$};
            \node[shape=circle,draw=black] (D) at (0,3) {$v_4$};

            \path (A) edge node {} (B);
            \path (A) edge node {} (C);
            \path (B) edge node {} (C);
            \path (A) edge node {} (D);
            \path (C) edge node {} (D);
            \path (B) edge node {} (D);

        \end{tikzpicture}
    \end{figure}
\end{ese}

\begin{defn}[grado di un vertice] Si chiama \emph{grado} di un vertice $v$ e si indica con $gr(v)$ il
numero di spigoli incidenti in $v$.
\end{defn}

\begin{ese}
	$gr(v_1) = 1$, \quad $gr(v_2) = 3$, \quad $gr(v_3) = gr(v_4) = gr(v_5) = 2$.
	\[\sum_{v \in V}^{} gr(v) = 1 + 3 + 2 + 2 + 2 = 10 = 2 |E| = 2 \times 5\]

	\begin{figure}[!ht]
	  	 \centering
	        \begin{tikzpicture}
		        \node[shape=circle,draw=black] (A) at (0,0) {$v_4$};
		        \node[shape=circle,draw=black] (B) at (3,0) {$v_5$};
		        \node[shape=circle,draw=black] (C) at (3,3) {$v_2$};
		        \node[shape=circle,draw=black] (D) at (0,3) {$v_3$};
		        \node[shape=circle,draw=black] (E) at (6,3) {$v_1$};

		        \path (C) edge node {} (E);
				\path (A) edge node {} (B);
		        \path (B) edge node {} (C);
		        \path (A) edge node {} (D);
		        \path (C) edge node {} (D);

	    	\end{tikzpicture}
	\end{figure}
\end{ese}

\begin{thm}
In ogni grafo semplice non orientato $G(V,E)$, la somma dei gradi di tutti i vertici è
uguale al doppio del numero degli spigoli.
\begin{equation} \label{eq:thm1} \sum_{v \in V}^{} gr(v) = 2|E| \end{equation}
\proof
	Per induzione su $m = |E|$:\\
	\emph{caso base: $m = 0$}\\	
	\indent $gr(v) = 0 \quad \forall v \in V$, \quad $|E| = 0$.\\
	\emph{passo induttivo:}\\
	\indent sia $G(V,E)$ un grafo con $m$ spigoli. Si suppone che ~\ref{eq:thm1} 
	sia valida $\forall$ grafo con $m-1$ spigoli.\\
	Siano $(\bar{u}, \bar{v}) \in E$ e $G'(V, E'=E \setminus \{(\bar{u}, \bar{v})\})$ ottenuto da $G$ togliendo 
	$(\bar{u}, \bar{v})$.\\
	Si può notare che $gr_G(\bar{u})= gr_{G'}(\bar{u})+1$, $gr_G(\bar{v}) = gr_{G'}(\bar{v})+1$ e quindi
	${\forall x \in V \text{, } x \neq \bar{u} \text{, }  x \neq \bar{v}: \text{ } gr_G(x) = gr_{G'}(x)}$.\\%https://tex.stackexchange.com/questions/10850/stop-latex-from-breaking-an-inline-math-equation
	$|E'| = |E|-1 = m -1 \implies$ in $G'$ vale l'ipotesi induttiva 
    ${\implies \sum_{v \in V}^{} gr_{G'}(v) = 2|E'|}$.\\In $G$:

    %https://it.sharelatex.com/learn/Aligning_equations_with_amsmath
    \begin{equation*}
    \begin{split}
	 \sum_{v \in V}^{} gr(v) & = 
     \sum_{v \in V \text{, } v \neq \bar{u} \text{, } v \neq \bar{v} }^{} 
        gr_G(v)+gr_G(\bar{u})+gr_G(\bar{v}) \\ & =
	 \sum_{v\in V\text{, }v\neq \bar{u}\text{, } v\neq \bar{v} }^{} 
        gr_{G'}(v)+gr_{G'}(\bar{u})+1 +gr_{G'}(\bar{v}) +1 \\ & =
	 \sum_{v \in V}^{} gr_{G'}(v) + 2 \underbrace{=}_{\text{ipotesi induttiva}} 2|E'| + 2 \\
     & = 2(m-1) + 2 \\
	 & = 2m \\ & = 2|E|
    \end{split}
    \end{equation*}
        
\endproof
\end{thm}

\begin{cor}
    In ogni grafo non orientato, il numero dei vertici di grado dispari è pari.
\proof
    Siano $G(V,E)$ un grafo non orientato semplice, ${V_d = \{v \in V \mid gr(v) \text{ è dispari}\}}$ e
    ${V_p = \{v \in V \mid gr(v) \text{ è pari}\}}$; quindi ${V_d \cap V_p = \emptyset}$ e 
    ${V_d \cup V_p = V}$.
    \begin{equation*}
    \begin{split}
        \sum_{v \in V}^{} gr(v) & = 
        2|E| \\ & = 
        \underbrace{{\sum_{v \in V_p}^{} gr(v)}}_{\text{pari}} + {\sum_{v \in V_d}^{} gr(v)} = 
        \underbrace{2|E|}_{\text{pari}}
    \end{split}
    \end{equation*}
    Poichè la somma dei gradi dei vertici che hanno grado pari è un numero pari, 
    allora anche la somma dei gradi dei vertici che hanno grado dispari è un numero pari perché:
    \[
        {\sum_{v \in V_d}^{} gr(v)} = \underbrace{2|E|}_{\text{pari}} - 
                \underbrace{{\sum_{v \in V_p}^{} gr(v)}}_{\text{pari}}
    \]
    e la differenza tra due numeri pari è un numero pari. Essendo quindi la somma dei gradi dei vertici
    di grado dispari un numero pari, allora anche il numero dei vertici di grado dispari è un numero
    pari. Questo perchè se si sommano $n$ numeri dispari, la loro somma è un numero pari 
    se e solo se $n$ è pari.
\endproof
\end{cor}

\begin{eser}
    Trovare $G(V,E)$ con $|V| = 7$ e ${gr(v) = 5 \text{ } \forall v \in V}$.\\
    Svolgimento: non esiste alcun grafo di questo tipo, per il corollario di cui sopra.%\QEDA
\end{eser}




\section{Prime proprietà dei grafi orientati}
Sia $G(V,A)$ un grafo orientato semplice. Allora il minimo numero di archi che questo può
avere è $0$ (ogni vertice è isolato) mentre il massimo è ${|V| \cdot (|V|-1)}$.

\begin{ese}
Due grafi orientati con il loro massimo numero di archi possibili.
    \begin{figure}[h]
    \centering
        \begin{tikzpicture}
            \node[shape=circle,draw=black] (A) at (0,0) {$v_1$};
            \node[shape=circle,draw=black] (B) at (2,0) {$v_2$};

            \path [->] (A) edge [bend left=20] node[right] {} (B);
            \path [->] (B) edge [bend left=20] node[right] {} (A);
        \end{tikzpicture}
        \hspace{1cm}
        \begin{tikzpicture}
            \node[shape=circle,draw=black] (A) at (0,0) {$v_1$};
            \node[shape=circle,draw=black] (B) at (-1.5, -2.5) {$v_2$};
            \node[shape=circle,draw=black] (C) at (1.5, -2.5) {$v_3$};

            \path [->] (A) edge [bend left=20] node[right] {} (B);
            \path [->] (A) edge [bend left=20] node[right] {} (C);
            \path [->] (B) edge [bend left=20] node[right] {} (C);
            \path [->] (B) edge [bend left=20] node[right] {} (A);
            \path [->] (C) edge [bend left=20] node[right] {} (B);
            \path [->] (C) edge [bend left=20] node[right] {} (A);
        \end{tikzpicture} 
        \caption{Due grafi orientati con il loro massimo numero di archi possibile}
       
    \end{figure}
\end{ese}

\begin{defn}[grado entrante di un vertice]
Si chiama grado \emph{entrante} di un vertice $v$ e si indica con ${In\text{-}deg(v)}$ il numero
di archi entranti nel vertice v.
\end{defn}

\begin{defn}[grado uscente di un vertice]
Si chiama grado \emph{uscente} di un vertice $v$ e si indica con ${Out\text{-}deg(v)}$ il numero
di archi uscenti dal vertice v.
\end{defn}

\begin{ese}
    La figura~\ref{fig:in_out} rappresenta un grafo con 
    ${In\text{-}deg(v_1) = 1}$ e ${Out\text{-}deg(v_1) = 3}$
    \begin{figure}[!ht]
        \centering
            \begin{tikzpicture} 
                \node[shape=circle,draw=red] (A) at (0,0) {$v_1$};
                \node[shape=circle,draw=black] (B) at (-3,-2) {$v_2$};
                \node[shape=circle,draw=black] (C) at (0,-2) {$v_3$};
                \node[shape=circle,draw=black] (D) at (3,-2) {$v_3$};


                \path [->, draw=green] (A) edge [bend right=20] node[right] {} (B);
                \path [->, draw=orange] (B) edge [bend right=20] node[right] {} (A);
                \path [->, draw=green] (A) edge node[right] {} (C);
                \path [->, draw=green] (A) edge [bend left=20] node[right] {} (D);
            \end{tikzpicture}
            \caption{esempio con ${In\text{-}deg(v_1) = 1}$ e ${Out\text{-}deg(v_1) = 3}$ } 
            \label{fig:in_out}
    \end{figure}
\end{ese}


\begin{thm}
In ogni grafo orientato $G(V,A)$ sono uguali tra loro: la somma dei gradi
uscenti dei nodi, la somma dei gradi entranti dei nodi, il numero di archi del grafo.
\[ {\sum_{v \in V}^{} In\text{-}deg(v)} = {\sum_{v \in V}^{} Out\text{-}deg(v)} = |A|\]

\proof
Da completare
\endproof
\end{thm}



\end{document}



\marginpar{⟨nota a margine⟩}
\footnote{⟨nota a piè di pagina⟩}
